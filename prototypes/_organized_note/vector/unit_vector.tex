\documentclass[12pt]{article}
\usepackage{emoji}
\usepackage{fontspec} % Required for font settings
\usepackage[utf8]{inputenc}

% Set the main font to Noto Sans CJK KR for Korean and English
\setmainfont{Noto Sans CJK KR}
\newfontfamily\emojiFont{Noto Color Emoji} % Optional: for explicit emoji handling

\usepackage{xcolor}   % Required for color settings
% Set background color and text color
\pagecolor{black}  % Set the background color to black
\color{white}      % Set the text color to white

\usepackage{amsmath}  % Required for mathematical symbols
\usepackage{graphicx} % Required for including images
\usepackage{tikz}     % 3D 그래프 그리기
\usepackage{pgfplots} % 3D 그래프 추가
\usepackage{hyperref} % 하이퍼링크 추가

% Set up hyperref options
\hypersetup{
    colorlinks=true,     % 하이퍼링크에 색상을 적용
    linkcolor=cyan,      % 목차와 섹션 링크의 색상
    urlcolor=cyan,       % URL 링크 색상
    pdftitle={단위벡터 (Unit Vector)},
    pdfauthor={Author: wbfw109v2},
    pdfsubject={벡터의 크기를 1로 만드는 기초 개념}
}

\title{
    단위벡터 (Unit Vector): \\
    벡터의 크기를 1로 만드는 기초 개념
}
\author{Author: wbfw109v2}
\date{\today}

\begin{document}

% Title and Table of Contents
\maketitle
\tableofcontents

\section{배경과 역사}

\noindent 단위벡터의 개념은 벡터 공간에서 크기를 1로 만들어 벡터의 방향을 강조하는 기초적인 수학적 정의입니다. 역사적으로, 단위벡터는 기하학적 문제에서 물체의 방향을 나타내기 위해 처음 도입되었으며, 이후 물리학과 공학에서 중요한 문제 해결 도구로 사용되어 왔습니다.

\vspace{1\baselineskip}
\noindent 물리학에서는 힘이나 속도의 방향을 나타낼 때 단위벡터를 사용하여, 벡터의 방향만을 강조할 수 있습니다. 이러한 방식은 복잡한 물리적 현상에서 불필요한 크기 요소를 제거하고, 순수한 방향 정보만을 남길 수 있게 해 줍니다.


\section{정의와 목적}

\noindent 단위벡터는 벡터의 크기를 1로 조정하여 방향만을 표현하는 벡터입니다. 벡터 \( \mathbf{v} \)의 단위벡터 \( \hat{\mathbf{v}} \)는 다음과 같이 정의됩니다:
\[
  \hat{\mathbf{v}} = \frac{\mathbf{v}}{|\mathbf{v}|}
\]
여기서 \( |\mathbf{v}| \)는 벡터의 크기입니다.

\noindent 단위벡터는 원래 벡터의 방향만을 남기면서, 그 크기를 1로 만들기 때문에 방향을 명확하게 표현하는 데 유용합니다. 이는 특히 컴퓨터 그래픽스와 같은 분야에서 물체의 표면 법선(Normal vector)을 정확하게 표현하는 데 자주 사용됩니다.


\section{연산의 이유와 유용성}

\noindent 단위벡터는 벡터의 방향을 유지하면서 크기를 1로 정규화합니다. 이렇게 정규화하는 이유는 계산의 일관성과 편리성을 높이기 위함입니다. 복잡한 벡터 연산에서 크기 대신 방향만을 유지함으로써, 연산의 복잡성을 줄이고 계산을 간단하게 할 수 있습니다.

\vspace{1\baselineskip}
\noindent \emoji{shopping} e.g. 벡터 \( \mathbf{v_1} = (3, 4) \)의 크기는 다음과 같이 계산됩니다:
\[
  |\mathbf{v_1}| = \sqrt{3^2 + 4^2} = 5
\]
따라서 단위벡터는 벡터 \( \mathbf{v_1} \)를 그 크기로 나누어 다음과 같이 계산됩니다:
\[
  \hat{\mathbf{v_1}} = \left( \frac{3}{5}, \frac{4}{5} \right)
\]
이 단위벡터는 원래 벡터의 방향을 유지하되, 크기는 1로 조정되었습니다. 이처럼 단위벡터를 통해 벡터의 방향만을 정확하게 표현할 수 있습니다.


\section{비교 | 대조 | 성질}

\noindent 단위벡터는 고유한 기하학적 성질을 가지고 있으며, 다른 벡터 연산과 차별화됩니다. 단위벡터는 크기가 1로 고정되므로, 오직 벡터의 방향만을 유지합니다. 반면에, 벡터의 내적이나 외적 연산은 벡터의 크기와 방향을 모두 고려합니다.

예를 들어, 벡터 내적은 다음과 같이 계산됩니다:
\[
  \mathbf{v_1} \cdot \mathbf{v_2} = |\mathbf{v_1}| |\mathbf{v_2}| \cos{\theta}
\]
이 식에서 벡터의 크기와 두 벡터 사이의 각도 \( \theta \)에 따른 코사인 값이 결합되어, 두 벡터의 평행 여부를 판단하는 데 사용됩니다.

\noindent 한편, 벡터 외적은 다음과 같이 계산됩니다:
\[
  \mathbf{v_1} \times \mathbf{v_2} = |\mathbf{v_1}| |\mathbf{v_2}| \sin{\theta} \hat{n}
\]
벡터 외적은 두 벡터가 만드는 평면에 수직인 벡터를 나타내며, 그 크기는 두 벡터가 이루는 평행사변형의 넓이에 해당합니다.


\section{응용 | 실제 사례}

\noindent 단위벡터는 다양한 실생활 응용에서 중요한 역할을 합니다.
\vspace{1\baselineskip}
\noindent \emoji{shopping} e.g. 물리학에서는 물체의 운동 방향을 나타낼 때 단위벡터를 사용합니다. 속도 벡터를 정규화함으로써, 물체가 이동하는 방향을 명확하게 표현할 수 있습니다.

\noindent 또한, 컴퓨터 그래픽스에서도 단위벡터는 중요한 도구입니다. 카메라의 시점이나 물체의 표면 법선을 표현할 때, 단위벡터를 통해 3D 공간에서의 물체의 시각적 표현을 정확하게 할 수 있습니다. 이는 3D 렌더링에서 물체의 표면이 어떻게 보이는지를 결정하는 데 매우 중요한 역할을 합니다.


\section{관련 논문 | 참고 자료}

\noindent 단위벡터와 벡터 연산에 대한 심도 있는 논문 및 연구는 선형대수학(Linear Algebra)과 벡터 공간 이론에서 찾아볼 수 있습니다. 이론적 기초를 더 깊이 이해하고자 한다면, 다음과 같은 참고 문헌을 확인해보세요.


\vspace{1\baselineskip}
\noindent \emoji{shopping} e.g. \textit{Gilbert Strang}, \textit{Linear Algebra and Its Applications}에서는 벡터의 정규화와 관련된 중요한 이론들을 다룹니다. 이 책은 벡터의 기초 이론부터 응용까지 포괄적으로 설명하고 있어 단위벡터의 활용을 더욱 깊이 이해하는 데 도움이 됩니다.

\end{document}
